%==============================
\subsection{Overall Analyses}\label{subsec:overall}


%==============================================
\subsubsection{Online Execution and Adaptation}\label{subsec:execute}

Once the hybrid pushing plan $\nu^\star$ is generated, the robot fleet executes the task
in real-time by following the guiding path $\overline{\mathbf{S}} = \{\mathbf{s}_0, \dots,
\mathbf{s}_L\}$. Each robot tracks the assigned trajectory segments, pushing the target
along the planned path while adjusting its motion based on the current target state
$\mathbf{s}_m(t)$ and interaction mode $\xi_n(t)$. The robots apply the desired forces
at the contact points $\mathbf{c}_n(t)$ using the control law:
\[
  \widehat{\mathbf{v}}_n = K_{\texttt{vel}} \left(\widehat{\mathbf{c}}_n - \mathbf{c}_n\right), \quad
  \widehat{\omega}_n = K_{\texttt{rot}} \left(\widehat{\psi}_n - \psi_n\right),
\]
to guide the object from the start to the goal while ensuring contact force consistency.
Transitioning between obstacles is handled according to the hybrid plan. Specifically,
as each robot pushes the object, it follows a **collision-free path** to the next set of
contact points on the next obstacle, ensuring that all movement respects the planned
interaction modes, and the sequence of obstacles is maintained. This transition is dynamically
adjusted as the robots continue pushing the target toward the goal.

In case of failures, such as when the object deviates from its expected position,
the algorithm triggers adaptation. If the object fails to reach the desired position,
the system checks the tracking error:
\[
  \mathrm{dist}(\mathbf{s}_m(t'), \varrho_\ell) \geq \delta_{\texttt{f}}, \quad
  \|\mathbf{s}_m(t') - \mathbf{s}_m(t'')\| < r_{\texttt{stuck}},
\]
and re-computes the entire hybrid plan based on the current system state. This includes
recalculating the sequence of movable obstacles to push, the contact points, and the
pushing forces to be applied. The updated plan is then executed with the control law in
\eqref{eq:control} to guide the robots toward the goal despite disruptions, ensuring
the task is completed even under unforeseen circumstances.



%==============================
\subsubsection{Complexity Analysis}\label{subsubsec:complexity}

The time complexity of the proposed method is dominated by three main components: the W-Clearance
Connectivity Graph (WCCG), gap-ranking strategy, and simulation-in-the-loop hybrid search. Constructing
the WCCG has a complexity of $\mathcal{O}(M^2 \log M)$, where $M$ is the number of movable obstacles,
due to the graph search and clearance checks. The gap-ranking strategy, which involves evaluating each
critical gap and ranking them based on their clearance and the obstacles’ mass, has a complexity of
$\mathcal{O}(G \log G + G M)$, where $G$ is the number of critical gaps and $M$ is the number of obstacles.
The most computationally intensive part is the simulation-in-the-loop hybrid search, which checks the feasibility
of each mode by simulating robot dynamics. The complexity for each simulation step is $\mathcal{O}(N^{3.5})$, where
$N$ is the number of robots, and the overall complexity of the hybrid search is $\mathcal{O}(N^{3.5} \cdot M \cdot L)$,
where $L$ is the number of keyframes in the path. Hence, the overall complexity is primarily determined by the hybrid
search algorithm.




%==============================
\subsubsection{Generalization}\label{subsec:general}

The proposed framework can be generalized in several directions. (I) \emph{Heterogeneous robots}:
When robots have varying capabilities, such as different maximum forces, the hybrid plan $\nu^\star$
should be adjusted by assigning more powerful robots to obstacles that require higher forces,
with smaller robots handling less demanding tasks. This adjustment is reflected in the interaction modes
$\boldsymbol{\xi} = (\mathbf{c}_1, \mathbf{f}_1, R_1), \dots, (\mathbf{c}_N, \mathbf{f}_N, R_N)$,
where $\mathbf{f}_n$ represents the force for each robot. (II) \emph{Simultaneous clearing}:
The framework can be extended to allow multiple obstacles to be pushed simultaneously,
improving the efficiency of path clearing. This requires modifying the WCCG, $\mathcal{G}^W(t)$,
to support parallel tasks and ensuring collision-free paths for multiple robots. (III) \emph{Dynamic
obstacle movement}: The method can handle dynamic obstacles by incorporating real-time sensor data
and updating the WCCG, allowing the hybrid search algorithm in Section~\ref{subsec:simloop} to adapt the plan
on-the-fly to avoid collisions and clear the path efficiently, even with moving obstacles.
