%==============================
\subsection{Overall Analyses}\label{subsec:overall}

\subsubsection{Execution and Online Adaptation}\label{subsec:execute}
The hybrid search outputs a pushing schedule
$\pi=\{\tau_1,\cdots,\tau_K\}$, where each $\tau_k$ specifies a target gap, a
short-horizon velocity $\mathbf{v}_k\triangleq(v_x,v_y,\omega)$, and a contact
mode $\boldsymbol{\xi}_k$. Execution proceeds sequentially under a hybrid
controller alternating between transition and pushing.

During transition, robots plan collision-free paths in the instantaneous
free space $\widehat{\mathcal{W}}(t)$ to reach contact points.
If two paths are on a collision course, the corresponding contacts are swapped to
avoid deadlock. During pushing, the velocity $\mathbf{v}_k$ is integrated to
generate a reference trajectory,
which is mapped to per-robot contact references using $\boldsymbol{\xi}_k$.
Each robot $R_n$ applies proportional
velocity control:
$\mathbf{v}_n = K_{\!p}(\widehat{\mathbf{p}}^{\,\text{c}}_n-\mathbf{p}^{\text{c}}_n)$,
$\omega_n = K_{\!r}(\widehat{\psi}^{\,\text{c}}_n-\psi^{\text{c}}_n)$,
where $(\widehat{\mathbf{p}}^{\,\text{c}}_n,\widehat{\psi}^{\,\text{c}}_n)$ are
reference states, $(\mathbf{p}^{\text{c}}_n,\psi^{\text{c}}_n)$ are measured
states, and $K_{\!p},K_{\!r}>0$ are control gains.
Offsets are adjusted online to counter yaw drift and maintain stable contact.

Execution is monitored by timeouts during transitions and early-stop tests in pushing.
If a push succeeds, execution advances to $\tau_{k+1}$; otherwise,
control is returned to the planner. Periodically, the WCCG is rebuilt and
execution terminates once a $W$--clear path $\mathcal{P}^W_\texttt{V}$ connects
$\mathbf{s}_\texttt{V}^{\texttt{S}}$ and $\mathbf{s}_\texttt{V}^{\texttt{G}}$.

\begin{remark}[Endpoint Refinement]
The $W$--CCG criterion may be conservative near endpoints. If connected
but the disks $\mathbb{B}_{W/2}(\mathbf{s}_\texttt{V}^{\texttt{S}})$ or
$\mathbb{B}_{W/2}(\mathbf{s}_\texttt{V}^{\texttt{G}})$ intersect obstacles, a
set of auxiliary pushes is applied to clear the disks and enable
execution. \hfill$\blacksquare$
\end{remark}


%----------------------------------------------
\subsubsection{Computational Complexity}\label{subsubsec:complexity}
The cost of each node expansion is dominated by simulation of pushing strategies. 
Construction of the WCCG $\mathcal{G}_W$ and the associated connectivity tests scales nearly linearly 
in the number of movable
obstacles $|\boldsymbol{\Omega}|$, while gap ranking by presearch
in Sec.~\ref{subsec:gap} is lightweight as it is limited to a small subset
of frontier gaps. At expansion, only a fraction of the strategies in
$\mathsf{Rank}(\nu)$ survive the geometric quick-pass and early-stop checks,
which shortens the horizon of the simulator calls defined by
$\mathsf{EvalSim}(\cdot)$. Parallel execution of these simulations across multiple
workers reduces the effective cost nearly linearly until communication overhead
is reached. Deferred expansion further improves efficiency by distributing the
evaluation of $\mathsf{Rank}(\nu)$ across iterations, avoiding redundant
restarts and keeping the priority queue focused on promising frontiers.


%----------------------------------------------
\subsubsection{Generalization}\label{subsec:general}
The framework admits several extensions: 
(I) \textit{Heterogeneous teams.} Per-robot gain tuning and robot-specific mode 
priors enable cooperation among diverse robots; 
(II) \textit{Concurrent pushing.} Multi-object mode generation and conflict-aware 
transition planning allow multiple obstacles to be pushed simultaneously; 
(III) \textit{Dynamic environments.} Periodic $W$--connectivity checks and 
on-demand replanning handle disturbances such as object drift or unmodeled 
contacts during execution.

