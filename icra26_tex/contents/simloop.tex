%==============================================
\subsection{Simulation-in-the-Loop Search}
\label{subsec:simloop}

The final step integrates physical feasibility directly into the planning
process through a simulation-in-the-loop search scheme. Unlike pure
graph-based reasoning, which abstracts away dynamics, this method validates
candidate pushing actions by embedding a fast parallelized physics engine
inside the search loop. In this way, both the sequence of obstacle pushes
and their dynamic feasibility are determined consistently, leading to robust
execution in realistic settings. The procedure consists of three phases:
(i) node selection based on a priority queue;
(ii) expansion through candidate pushing actions validated in simulation;
(iii) termination upon reaching a feasible clearance-$W$ path.

%==============================
\subsubsection{Node Representation and Cost}
\label{subsubsec:simloop-node}
Each search node represents the current configuration of movable obstacles
together with the path status between start and goal. Formally, let
$\nu=(\boldsymbol{\Omega}(t),\,\mathcal{P}_W(t))$, where
$\boldsymbol{\Omega}(t)$ is the current set of obstacle states and
$\mathcal{P}_W(t)$ denotes the existence of a clearance-$W$ path in the
WCCG. The cost of a node is defined as
\begin{equation}\label{eq:simloop-cost}
  \mathrm{C}(\nu)\triangleq T(\nu)+\alpha
  \sum_{g\in\mathcal{G}^{\texttt{crit}}(t)} \mathsf{C}(g,t),
\end{equation}
where $T(\nu)$ is the elapsed task duration,
$\mathcal{G}^{\texttt{crit}}(t)$ is the set of critical gaps, and
$\mathsf{C}(g,t)$ is the clearing cost defined in~\eqref{eq:gap-cost}.
This balances planning time against clearing effort.

%==============================
\subsubsection{Simulation-Guided Expansion}
\label{subsubsec:simloop-expansion}
When expanding a node, the planner hypothesizes pushing one of the blocking
obstacles adjacent to a critical gap. Let $\xi=(\mathbf{c},\mathbf{f})$
denote a candidate pushing mode, with contact point $\mathbf{c}\in\partial\Omega_m$
and force $\mathbf{f}\in\mathbb{R}^2$ applied by a subgroup of robots. The
candidate action is validated through a short-horizon simulation
\begin{equation}\label{eq:simloop-dynamics}
  \Omega_m(t+\Delta t) \leftarrow
  \texttt{SimDyn}\big(\Omega_m(t),\xi,\Delta t\big),
\end{equation}
where $\texttt{SimDyn}(\cdot)$ integrates the coupled robot-object dynamics
with contact forces and friction. If the simulated trajectory avoids
collisions and improves the clearance of at least one critical gap, the
resulting state $\boldsymbol{\Omega}(t+\Delta t)$ is inserted into the queue
as a new node.

%==============================
\begin{algorithm}[t]
  \caption{Simulation-in-the-Loop Search (SiLS).}
  \label{alg:SiLS}
  \SetAlgoLined
  \KwIn{Initial obstacle states $\boldsymbol{\Omega}(0)$, start $\mathbf{s}^{\texttt{S}}$, goal $\mathbf{s}^{\texttt{G}}$.}
  \KwOut{Feasible clearance-$W$ plan $\nu^\star$.}
  Initialize priority queue $Q=\{\nu_0\}$ with $\nu_0=(\boldsymbol{\Omega}(0),\emptyset)$;\\
  \While{$Q\neq\varnothing$}{
    \tcc{\textbf{Selection}}
    Pop $\nu^\star=\textbf{argmin}_{\nu\in Q}\{\mathrm{C}(\nu)+\mathrm{H}(\nu)\}$;\\
    Check path existence in $\mathcal{G}^W(t)$; if feasible, return $\nu^\star$;\\
    \tcc{\textbf{Expansion}}
    Identify $\mathcal{G}^{\texttt{crit}}(t)$ from WCCG;\\
    \ForEach{$g\in\mathcal{G}^{\texttt{crit}}(t)$}{
      Generate candidate pushing modes $\xi\in\Xi(g)$;\\
      Simulate next state $\boldsymbol{\Omega}(t+\Delta t)=\texttt{SimDyn}(\Omega_m,\xi,\Delta t)$;\\
      \If{state is collision-free and improves $w(g,t)$}{
        Insert new node $\nu^+$ into $Q$;\\
      }
    }
  }
  Return best $\nu^\star$ if time limit reached;
\end{algorithm}
%==============================

%==============================
\subsubsection{Correctness and Completeness}
\label{subsubsec:simloop-theory}
The simulation-in-the-loop search inherits both correctness and probabilistic
completeness under mild assumptions. First, correctness follows since each
expansion is validated by a dynamics simulator that enforces physical
constraints, guaranteeing that any returned plan is dynamically feasible.

\begin{lemma}[Correctness]\label{lemma:correctness}
Any plan $\nu^\star$ returned by Algorithm~\ref{alg:SiLS} is feasible for the
physical system, as every action has been validated through forward
simulation with collision checking.
\end{lemma}

Completeness is ensured by the hybrid search structure: as long as the set
of pushing modes $\Xi$ is sufficiently rich and the simulation horizon
$\Delta t$ is bounded below, all feasible solutions are eventually explored.

\begin{theorem}[Probabilistic Completeness]\label{thm:completeness-simloop}
  If a clearance-$W$ path exists, Algorithm~\ref{alg:SiLS} finds a feasible
  plan $\nu^\star$ with probability approaching $1$ as the number of node
  expansions tends to infinity.
\end{theorem}

The proofs follow by extending standard arguments from randomized search
algorithms with embedded feasibility oracles, noting that the simulator
acts as a physical feasibility oracle in this case.
