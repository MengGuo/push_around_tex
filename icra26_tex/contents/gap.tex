%==============================================
\subsection{Gap Ranking Strategy}
\label{subsec:gap}

When no clearance-$W$ path exists, the critical gaps identified from the
WCCG must be prioritized for obstacle-clearing actions. Since multiple gaps
may block connectivity, it is necessary to rank them according to the
estimated cost of clearing, so that robots can focus on the most promising
ones. The gap-ranking strategy assigns a heuristic score to each gap by
considering both the physical effort required to widen it and the expected
benefit of restoring connectivity.

%==============================
\subsubsection{Gap Clearing Cost}
\label{subsubsec:gap-cost}
The first element of the ranking strategy is to estimate the physical
difficulty of widening a gap. Intuitively, narrow gaps require more
clearance, and heavy obstacles are harder to move. Thus, the cost of
clearing should depend on both the clearance deficiency and the blocking
objects. For a gap~$g\in\mathcal{G}^{\texttt{crit}}(t)$ with width
$w(g,t)<W$, the additional clearance required is defined as
$\Delta w(g,t)\triangleq W-w(g,t)$. Let
$\mathcal{O}(g,t)\subseteq\boldsymbol{\Omega}(t)$ denote the subset of
movable obstacles adjacent to $g$. The clearing cost is then formulated as
\begin{equation}\label{eq:gap-cost}
  \mathsf{C}(g,t)\triangleq \beta_1 \Delta w(g,t)
  + \beta_2 \sum_{\Omega_m\in\mathcal{O}(g,t)} \mathsf{M}_m,
\end{equation}
where $\beta_1,\beta_2>0$ are weighting parameters, and $\mathsf{M}_m$ is
the mass of obstacle $\Omega_m$. The first term measures the geometric
effort required to widen the gap, while the second reflects the physical
effort of moving adjacent obstacles.

%==============================
\subsubsection{Gap Ranking Rule}
\label{subsubsec:gap-ranking-rule}
The second element of the strategy is to evaluate how much clearing a
specific gap improves connectivity between the start and the goal. If a
gap is cleared, the effective shortest path through the WCCG may become
shorter or even newly feasible. Denote the current shortest-path length
between $v^{\texttt{S}}$ and $v^{\texttt{G}}$ by $d^{\texttt{cur}}(t)$
(possibly infinite if no path exists). Suppose gap~$g$ is hypothetically
cleared, yielding a modified graph $\mathcal{G}^{W}_{+g}(t)$ with updated
shortest-path length $d^{+g}(t)$. The connectivity benefit of clearing
gap~$g$ is then defined as
\begin{equation}\label{eq:gap-benefit}
  \mathsf{H}(g,t)\triangleq
  d^{\texttt{cur}}(t) - d^{+g}(t),
\end{equation}
where $\mathsf{H}(g,t)>0$ indicates a reduction in shortest-path length
after clearing $g$.

Finally, the ranking score combines benefit and cost into a single measure:
\begin{equation}\label{eq:gap-score}
  \mathsf{R}(g,t)\triangleq
  \frac{\mathsf{H}(g,t)}{\mathsf{C}(g,t)+\varepsilon},
\end{equation}
where $\varepsilon>0$ is a small constant for numerical stability. Gaps
with larger $\mathsf{R}(g,t)$ yield greater connectivity improvement per
unit of clearing effort, and are therefore prioritized for obstacle-clearing
actions in the next step.
