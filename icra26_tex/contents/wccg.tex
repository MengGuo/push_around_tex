%==============================================
\subsection{W-Clearance Connectivity Graph (WCCG)}
\label{subsec:wccg}

To reason about the existence of a traversable path of minimum width~$W$,
a W-Clearance Connectivity Graph (WCCG) is constructed. The key idea is to
represent the free space~$\widehat{\mathcal{W}}(t)$ as a graph whose edges
encode feasible passages with clearance at least~$W$, and whose nodes
correspond to connected free regions. This allows efficient connectivity
queries between start~$\mathbf{s}^{\texttt{S}}$ and goal~$\mathbf{s}^{\texttt{G}}$.

%==============================
\subsubsection{Free-Space Decomposition}
\label{subsubsec:wccg-decomposition}
The first step is to describe the free space available for navigation when
movable and static obstacles are present. Since obstacles may be arbitrarily
positioned, it is necessary to explicitly exclude their occupied areas to
obtain the traversable region. At time~$t$, the set of movable obstacles is
denoted by $\boldsymbol{\Omega}(t)\triangleq \{\Omega_1(t),\ldots,\Omega_M(t)\}$,
and the instantaneous free space is defined as
\begin{equation}\label{eq:free-space}
  \widehat{\mathcal{W}}(t) \triangleq
  \mathcal{W}\backslash
  \big(\mathcal{O}^{\texttt{fix}}\cup \{\Omega_m(t)\}_{m\in\mathcal{M}}\big),
\end{equation}
where $\mathcal{O}^{\texttt{fix}}$ are static obstacles and
$\Omega_m(t)$ are movable obstacles.

However, this free space does not yet guarantee that a large vehicle of width~$W$
can traverse through it. To capture this requirement, the free space is eroded
by a disk of radius~$W/2$, so that any valid path maintains clearance of at
least~$W$. The resulting clearance-valid free space is
\begin{equation}\label{eq:clearance-space}
  \widehat{\mathcal{W}}^{W}(t)\triangleq
  \{p\in\widehat{\mathcal{W}}(t)\,\mid\,
  \texttt{dist}(p,\partial\widehat{\mathcal{W}}(t))\geq W/2\},
\end{equation}
where $\texttt{dist}(p,\partial\widehat{\mathcal{W}}(t))$ is the Euclidean
distance from point~$p$ to the nearest boundary of $\widehat{\mathcal{W}}(t)$.

%==============================
\subsubsection{Graph Construction and Connectivity}
\label{subsubsec:wccg-construction}
Once the clearance-valid free space is obtained, it is represented in
graph form to facilitate connectivity analysis. Each connected region of
$\widehat{\mathcal{W}}^{W}(t)$ is treated as a node, and narrow gaps between
regions are modeled as edges if they are wide enough for traversal. The
WCCG is thus defined as
\begin{equation}\label{eq:wccg-def}
  \mathcal{G}^{W}(t)\triangleq \big(\mathcal{V}^{W}(t),\mathcal{E}^{W}(t)\big),
\end{equation}
where $\mathcal{V}^{W}(t)$ is the set of nodes representing connected
components of $\widehat{\mathcal{W}}^{W}(t)$, and $\mathcal{E}^{W}(t)$ is
the set of edges corresponding to separating gaps. Each gap is denoted
by $g$ with geometric width $w(g,t)\in\mathbb{R}_{+}$, and an edge exists
between two nodes if $w(g,t)\geq W$.

Using this graph, the existence of a path from the start to the goal can be
reduced to a standard connectivity check. Denote by~$v^{\texttt{S}}, v^{\texttt{G}}\in\mathcal{V}^{W}(t)$
the nodes containing $\mathbf{s}^{\texttt{S}}$ and $\mathbf{s}^{\texttt{G}}$, respectively.
Then a feasible path exists if and only if
\begin{equation}\label{eq:wccg-path}
  \exists \;\mathcal{P}_{W}\subseteq\mathcal{G}^{W}(t)
  \quad \text{s.t.}\quad v^{\texttt{S}}\leadsto v^{\texttt{G}},
\end{equation}
where $\mathcal{P}_{W}$ denotes a connected subgraph of $\mathcal{G}^{W}(t)$
linking start and goal nodes. When no such path exists, the blocking frontier
gaps are collected in the critical gap set
\begin{equation}\label{eq:critical-gaps}
  \mathcal{G}^{\texttt{crit}}(t)\triangleq
  \{g\in\mathcal{G}(t)\,\mid\,w(g,t)<W,\;
  g\;\text{lies on all paths between}\;v^{\texttt{S}},v^{\texttt{G}}\},
\end{equation}
where $\mathcal{G}^{\texttt{crit}}(t)$ contains exactly the narrow gaps that
must be widened by pushing nearby obstacles, and thus provides the natural
targets for subsequent ranking and obstacle-clearing steps.
