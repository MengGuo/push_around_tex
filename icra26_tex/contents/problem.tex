\section{Problem Description}\label{sec:problem}

%==============================
\subsection{Model of Workspace and Robots}\label{subsec:ws}
We consider a 2D workspace $\mathcal{W}\subset\mathbb{R}^2$ cluttered with
immovable obstacles $\mathcal{O}^{\texttt{fix}}$ and a set of $M$ movable rigid
polygons $\boldsymbol{\Omega}\triangleq\{\Omega_1,\ldots,\Omega_M\}\subset\mathcal{W}$.
Each $\Omega_m$ has mass $\mathsf{M}_m$, inertia $\mathsf{I}_m$, frictional
parameters (identified or estimated), state
$\mathbf{s}_m(t)\triangleq(\mathbf{x}_m(t),\psi_m(t))$, and occupied region
$\Omega_m(t)$.
A small, fixed team of robots $\mathcal{R}_{\mathrm{grp}}\subseteq\mathcal{R}$
(with $|\mathcal{R}_{\mathrm{grp}}|\in\{2,3\}$ in our experiments) operates as a
single cooperative unit; each robot $R$ has state
$\mathbf{s}_R(t)\triangleq(\mathbf{x}_R(t),\psi_R(t))$ and footprint $R(t)$.
The instantaneous free space is
\[
\widehat{\mathcal{W}}(t)\triangleq\mathcal{W}\setminus\Big(
\mathcal{O}^{\texttt{fix}}\cup\{\Omega_m(t)\}_{m=1}^{M}\cup\{R(t)\}_{R\in\mathcal{R}_{\mathrm{grp}}}
\Big).
\]
The “large payload” (or agent to be routed) is modeled as a disc of radius $W/2$;
a curve is $W$–feasible if its clearance is at least $W$ everywhere.

%==============================
\subsection{Collaborative Pushing Modes}\label{ss:interaction_mode}
Robots interact with a movable obstacle $\Omega_m$ through \emph{pushing modes}.
Since $\mathcal{R}_{\mathrm{grp}}$ acts as a single unit, a mode for $\Omega_m$
is specified as
$\boldsymbol{\xi}_m\triangleq(\mathcal{C}_m,\mathbf{u}_m)$,
where $\mathcal{C}_m\subset\partial\Omega_m$ are the contact locations realized
by the group and $\mathbf{u}_m$ encodes the nominal push action
(e.g., body–frame velocity or an equivalent wrench profile).
Let $\Xi_m$ denote the admissible mode set determined by contact geometry and
frictional limits. Different modes induce different motions of $\Omega_m$
through the physics engine.

%==============================
\subsection{Problem Statement}\label{subsec:objective}
Given start $\mathbf{s}^{\texttt{S}}$ and goal $\mathbf{s}^{\texttt{G}}$,
the objective is to compute a \emph{hybrid schedule}
$\pi=\{(m_k,\boldsymbol{\xi}_k,\Delta t_k)\}_{k=1}^{K}$ that reconfigures
$\{\Omega_m\}$ by sequential pushes of the robot team so that a $W$–feasible
path exists from $\mathbf{s}^{\texttt{S}}$ to $\mathbf{s}^{\texttt{G}}$.
Let $T\triangleq\sum_{k=1}^{K}\Delta t_k$ be the task duration, and
$J(m_k,\boldsymbol{\xi}_k;\mathbf{S}(\tau_k))$ the physics-based effort/feasibility
cost of executing mode $\boldsymbol{\xi}_k\in\Xi_{m_k}$ starting at system state
$\mathbf{S}(\tau_k)$.
We seek a compact single-column form:
\vspace{-2pt}
\begin{equation}\label{eq:problem}
\begin{aligned}
&\min_{\{(m_k,\boldsymbol{\xi}_k,\Delta t_k)\}_{k=1}^{K}}
\quad  T\;+\;\alpha\sum_{k=1}^{K} J\!\left(m_k,\boldsymbol{\xi}_k;\mathbf{S}(\tau_k)\right) \\
& \text{s.t.}\quad
\boldsymbol{\xi}_k\in\Xi_{m_k},\ \Delta t_k>0,\ \tau_{k+1}=\tau_k+\Delta t_k, \\
& \ \quad\quad \mathbf{S}(t^+)=\Phi\big(\mathbf{S}(t),\,m_k,\,\boldsymbol{\xi}_k\big),\ t\in[\tau_k,\tau_{k+1}), \\
& \ \quad\quad R(t)\cap R'(t)=\emptyset,\ \Omega_i(t)\cap\Omega_j(t)=\emptyset,\ \forall t, \\
& \ \quad\quad R(t)\cap\Omega_m(t)=\emptyset,\ R(t),\Omega_m(t)\subset\widehat{\mathcal{W}}(t),\ \forall t, \\
& \ \quad\quad \exists\,\mathcal{P}_W\subset\widehat{\mathcal{W}}(T):\ S\leadsto G,\ \text{clr}(\mathcal{P}_W)\ge W.
\end{aligned}
\end{equation}
\vspace{-4pt}

Here, $\Phi$ is the (simulator-consistent) state transition under the selected
pushing mode; $\alpha>0$ balances duration and effort. The schedule $\pi$
implicitly encodes \emph{which obstacle is pushed, how, and for how long};
no per-robot assignment variables are introduced, consistent with our use of a
fixed small team executing one push at a time.
