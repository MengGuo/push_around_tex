% ============================================================
\section{Numerical Experiments}
\label{sec:experiments}
We evaluate the proposed simulation-in-the-loop NAMO planner (SiLS)
in cluttered environments with both movable and immovable obstacles.
All components (W-CCG presearch, frontier extraction, and ModeTable prior)
are integrated as described in Sec.~\ref{sec:solution}.
The implementation is in \texttt{Python~3} and simulations are run in
\texttt{PyBullet}~\cite{coumans2019} on a laptop with an Intel
Core i7\textendash1280P CPU. Videos and logs are provided in the
supplementary material.

\begin{figure*}[t!]
  \centering
  \includegraphics[width=1\linewidth]{figures/sim_exp.png}
  \vspace{-2mm}
\caption{Execution--planning alignment in the main scenario. 
\textbf{Top:} five PyBullet snapshots of the scene. 
\textbf{Bottom:} corresponding WCCG with the start face highlighted as a translucent blue polygon.
The task moves a cylindrical object from the start \(S=(1,4)\) to the goal \(G=(7,2)\). 
In the final snapshot, the start face contains \(G\), 
consistent with the green execution trajectory in the top row, 
confirming successful completion from \(S\) to \(G\). 
% \textbf{Performance:} total execution time \(42.275\,\mathrm{s}\); 
% planning time \(7.437\,\mathrm{s}\) (1 iteration); expanded nodes \(8\); 
% visited nodes \(72\); simulations \(176\); push tasks executed \(6\); 
% latest WCCG connected \(=\) \texttt{True}.
}
\end{figure*}


% ========================================
\subsection{Numerical Simulations}
\label{subsec:sim}
% ========================================

\subsubsection{\change{Setup}}
\label{subsec:sim-setup}
\change{
The physics step is $\Delta t = 1/120$\,s and the control period is
$1/40$\,s. Robots are disk/box pushers with risk radii consistent with
the $W$-clearance definition. Movable obstacles have masses uniformly
sampled in $[5,15]$\,kg; immovables are modeled with mass $0$.
A trial succeeds when a $W$-clear path from start to goal exists and the
target reaches its goal disc.
}

\textbf{Workspace and scenarios.}
We use a \emph{nominal scenario} with an $8{\times}5$\,m bounded
workspace, two internal bars (bottlenecks), and a mixed pool of movable
shapes (curved and polygonal, including rings, ellipses, X/T/L/diamond,
arrow-like, rectangles, and cylinders). Two robots start in the lower
left; the target goal lies on the right side. Movables are randomly
placed with a minimum separation. We test \(10\text{--}15\) objects
over multiple seeds.

\textbf{Geometry and caching.}
Collision checks use polygon/curve-edge models with convex
decomposition when needed. Ray--edge queries for frontiers are
vectorized with AABB culling. Frontier loops are cached by signatures and
transitions $(\mathcal{L},g)\!\mapsto\!\mathcal{L}'$ are memoized.

\subsubsection{\change{Algorithm Configuration}}
\label{subsec:algo-config}
\change{
Unless stated otherwise: per-task simulation horizon $=80$ steps,
up to $64$ candidate push tasks per expansion, and priority
$f = g + \widehat{\mathsf{Cost}}_{\text{to-go}}$ with heuristic factor $10$.
Gap sampling uses a softmax with temperature $0.05$.
ModeTable is enabled (auto-baked if missing).
A \emph{quick-pass} geometry screen may skip physics if a reference
rollout already clears the gap; otherwise a short-horizon simulation with
early stop is used. To avoid premature termination, a
\emph{deferred-reinsertion} rule keeps high-value but temporarily
unexpanded nodes in the queue and revisits them later.
}
\begin{figure}[t!]
  \centering
  \includegraphics[width=\columnwidth]{figures/SL_Push.pdf}
  \vspace{-10mm}
\caption{SL-Push. Compute(offline) or simulate(sim-in-the-loop) the pushing directions for movable obstacles along the W-width straight path from sub-start to sub-goal. Top: Pushing steps. Bottom: Reachable region changes during pushing}
\label{fig:SL-Push}
\end{figure}

\begin{figure}[t!]
  \centering
  \vspace{-3mm}
  \includegraphics[width=\columnwidth]{figures/Rec_NAMO.pdf}
  \vspace{-10mm}
\caption{Rec-NAMO. Rec-NAMO builds path segments sequentially but fails to construct the full W-clear path from start to goal. Top: Pushing steps. Bottom: Reachable region changes during pushing}
\label{fig:Rec-NAMO}
\end{figure}


\subsubsection{\change{Baselines}}
\label{subsec:baselines}
We compare against three representative families, all using the same
$W$-clearance criterion and contact models:

\textbf{DFS-WCCG:} Simulation-in-the-loop depth-first search 
that shares our physics predictor and $W$–CCG for goal checking.
Each node is a snapshot; actions are four fixed axis-aligned pushes 
${!\leftarrow,\rightarrow,\uparrow,\downarrow!}$ 
applied to any movable object (branching $\le 4n$ for $n$ objects).

\textbf{SL-Push:} Straight-line (or waypointed) route; blockers are
cleared by (i) off-line minimal normal displacements or
(ii) sim-in-the-loop normal pushes from near to far.

\textbf{Rec-NAMO:} Recursive routing/pushing on a cost-weighted
  visibility graph: Dijkstra for routing, push-decomposition for local
  clearing; failures prune edges and replan.

\begin{figure*}
  \centering
  \includegraphics[width=1\linewidth]{figures/hardware_wccg.png}
  \vspace{-2mm}
\caption{Real-world pushing experiment with execution--planning alignment. 
\textbf{Top:} six snapshots from the physical scene as two robots push a target object across a \(5\times 6\,\mathrm{m}\) workspace with both movable pieces and fixed boundary segments. \textbf{Bottom:} corresponding WCCG overlays, where the start face is highlighted as a translucent blue polygon. Planning is posed with clearance \(W=1.0\), start \(S=(1.65,\,1.01)\), and goal \(G=(3.4,\,4.8)\). Across snapshots, the graph structure and candidate gaps evolve consistently with the observed object motion, illustrating that the planned corridors remain compatible with the real execution.}
\end{figure*}

\begin{figure}
  \centering
  \includegraphics[width=1\linewidth]{figures/hardware_snap.png}
  \vspace{-2mm}
\caption{Two-frame highlight of chain pushing in the real-world experiment. \textbf{Top:} two consecutive video snapshots illustrating a contact-induced cascade: Robot~R1 pushes a movable object, which in turn imparts motion to a neighboring object (chain push). \textbf{Bottom:} control panel showing commanded velocities for both robots (\(v_x\), \(v_y\), \(\omega\)). The temporal alignment between the frames and the plots reveals the contact transition: a brief rise in translational speed and differential angular commands near the interaction, followed by moderated controls as the secondary object begins to move. This pairing demonstrates that the planned interaction is executed as intended by both robots.}
\end{figure}

\subsubsection{\change{Metrics}}
\label{subsec:metrics}
We report success rate, wall-clock time, number of node expansions,
number of simulated pushes, cumulative push time, average presearch
cost-to-go, quick-pass ratio, early-stop ratio, and worker-pool
utilization. Results are averaged over 10 seeds unless noted.

% ==============================
\subsubsection{Comparison of Main Results}
\label{subsec:comparison}
Table~\ref{tab:main} summarizes the performance of four baseline methods and our proposed SiLS method on the nominal scenario. DFS-WCCG exhibits high computational overhead due to the exponential growth of the search space with the number of movable obstacles, resulting in very long planning times and frequent timeouts, which is reflected in its lowest success rate. SL-Push (offline) achieves the fastest planning times, typically under 0.1 s, but the lack of physics-informed simulation leads to physically infeasible solutions. The simulation-in-the-loop type of SL-Push improves solution feasibility; however, it still requires many simulation calls, resulting in increased planning times. Furthermore, the straight-line pushing strategy may generate unnecessarily long action sequences, as the direct path is not always the minimal pushing path. Figure~\ref{fig:SL-Push} illustrates the step-by-step execution of SL-Push with two manually defined subgoals, showing how the environment gradually becomes connected such that the start and goal are in the same connected face.

Rec-NAMO, while faster than simulation-in-the-loop approaches and effective in classical NAMO tasks, struggles to generate fully connected W-clear paths in our setting. Although its recursive search produces locally connected regions along the pushing sequence, the final map often fails to contain a continuous W-clear path, as demonstrated in Figure~\ref{fig:Rec-NAMO}. This limitation is reflected in its relatively low success rate in Table~\ref{tab:main}.

Our SiLS method attains higher success rates with fewer simulation calls and shorter planning and execution times, thanks to (i) frontier-based gap ranking, (ii) ModeTable-guided push directions, and (iii) quick-pass/early-stop. In addition, deferred reinsertion prevents priority-queue starvation by revisiting previously generated high-value nodes when a batch of actions fails, further improving efficiency and solution quality.

% \subsubsection{Main Results}
% \label{subsec:main-results}
% Table~\ref{tab:main} summarizes performance on the nominal scenario.
% SiLS attains higher success with fewer simulations and lower time due to:
% (i) frontier-based gap ranking, (ii) ModeTable-guided push directions,
% and (iii) quick-pass/early-stop. Deferred reinsertion prevents
% priority-queue starvation by revisiting previously generated high-value
% nodes when a batch of actions fails.

\begin{table}[t]
  \centering
  \begin{threeparttable}
  \caption{Performance on the nominal scenario with push-count (mean $\pm$ std).}
  \label{tab:main}
  \vspace{2pt}
  
  % 🔑 在这里调节列间距
  \setlength{\tabcolsep}{2.3pt} % 默认是 6pt,调小
  
\begin{tabular}{lccccc}
\toprule
Method & Succ.~(\%) & \#PT (s) & \#ET (s) & \#Sims & \#Pushes \\
\midrule
DFS-WCCG            & 25.0  & 41.8  & N/A     & --    & --   \\
SL-Push (off-line)  & 62.5  & 0.03  & 145.1   & 0.0   & 7.0  \\
SL-Push (sim)       & 75.0  & 25.1  & 246.8   & 16.0  & 8.0  \\
Rec-NAMO            & 37.5  & 10.5  & 179.3   & 0.0   & 7.0  \\
\textbf{SiLS (ours)}& \textbf{92.5} & \textbf{10.3} & \textbf{28.6} & \textbf{121.5} & \textbf{6.0} \\
\bottomrule
\end{tabular}
  
  \begin{tablenotes}[flushleft]\footnotesize
  \item \textbf{Metrics.} \emph{Succ.} = success rate; \emph{\#PT} = planning time; \emph{\#ET} = execution time;
  \emph{\#Sims} = simulations invoked; \emph{\#Pushes} = length of executed push sequence.
  \end{tablenotes}
  \end{threeparttable}
  \end{table}
  


\subsubsection{Ablations}
\label{subsec:ablations}
We remove one component at a time: (i) W-CCG presearch (random gap order),
(ii) ModeTable prior (fallback “away\,+\,jitter” only),
(iii) quick-pass/early-stop (always full physics),
(iv) deferred reinsertion (terminate on empty queue).
Table~\ref{tab:ablation} shows consistent drops in success and increased
time/\#Sims when any component is disabled.

\begin{table}[t]
  \centering
  \caption{Ablations on the nominal scenario (averaged over 40 runs).}
  \label{tab:ablation}
  \vspace{2pt}

  {%
  \setlength{\tabcolsep}{3pt}
  \renewcommand{\arraystretch}{0.9}

  \begin{tabular}{lcccc}
  \toprule
  Variant & Succ.~(\%) & Time (s) & \#Sims & Quick-pass (\%) \\
  \midrule
  SiLS (full)            & \textbf{92.5} & \textbf{28.6} & \textbf{122} & \textbf{57.5} \\
  \ \ w/o presearch      & 85.0          & 35.2          & 178          & 25.0          \\
  \ \ w/o ModeTable      & 80.0          & 33.4          & 164          & 40.0          \\
  \ \ w/o quick-pass/ES  & 90.0          & 41.9          & 208          & 0.0           \\
  \ \ w/o reinsertion    & 82.5          & 31.1          & 151          & 47.5          \\
  \bottomrule
  \end{tabular}
  }%

  \vspace{3pt}
  \small
\end{table}
  


\subsubsection{Efficiency and Scalability}
\label{subsec:eff}
\textbf{Frontier caching} makes BugPlanner near-linear in visible edges.
\textbf{Parallel prediction} uses a persistent worker pool with per-round
broadcast of snapshots and reachable-contact sets to reduce IPC.
\textbf{Quick-pass} and \textbf{early-stop} skip or truncate physics when
geometry suffices. \textbf{Deferred reinsertion} maintains steady depth
growth even when many pushes are rejected.

\subsubsection{Qualitative Results}
\label{subsec:qual}
Figure~\ref{fig:qual} shows a typical run: frontiers and ranked gaps,
a top-ranked gap sequence, simulated pushes that widen bottlenecks, and
the final $W$-clearance path. Per-step overlays and GIFs are produced by
a lightweight snapshot logger.

% \begin{figure}[t]
% \centering
% \includegraphics[width=0.98\linewidth]{figures/qual_demo.pdf}
% \caption{Representative run: frontier (blue), ranked gaps (color-coded by
% predicted cost), simulated pushes (red), and the resulting
% $W$-clearance path (green).}
% \label{fig:qual}
% \end{figure}

\subsubsection{Reproducibility}
\label{subsec:repro}
Random seeds are fixed, JSONL snapshots are logged, and the driver and
scenario generator are released. ModeTable entries are auto-generated if
absent to ensure repeatability across machines.

